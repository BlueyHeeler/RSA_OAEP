\documentclass{article}

\title{Trabalho de Implementação 3 - Seminário \\
        Segurança Computacional}

\author{Davi Bueno e Erick Taira}
\date{28 de Janeiro de 2025}

\begin{document}

\maketitle

\section*{Como a inteligência artificial é utilizada para a educação em computação?}

\section{Introdução}
No mundo de hoje, dominado pela tecnologia, a computação se tornou uma habilidade crucial para o sucesso, tanto na vida pessoal quanto profissional. A educação em computação, responsável por equipar as novas gerações com as ferramentas e conhecimentos para navegar nesse universo digital, enfrenta o desafio de acompanhar o ritmo acelerado das inovações. É nesse cenário que a inteligência artificial (IA) surge como uma aliada para aprimorar e transformar os processos de ensino e aprendizagem. Este estudo propõe explorar as diversas maneiras pelas quais essa tecnologia está sendo utilizada para melhorar o ensino na área de computação. Através de uma análise abrangente da literatura existente, buscamos identificar as tendências, os desafios e as oportunidades que a IA apresenta para o futuro da educação em computação.

\section{Objetivo do estudo}
Esperamos que este estudo contribua para uma melhor compreensão das diversas aplicações da IA na educação em computação, fornecendo insights valiosos para educadores, pesquisadores e formuladores de políticas. Este trabalho busca contribuir para o desenvolvimento de estratégias eficazes para integrar essa tecnologia na educação em computação.

\section{Artigo escolhido:}

The Use of Artificial Intelligence (AI) in Online Learning and Distance Education Processes: A Systematic Review of Empirical Studies

Segundo os critérios de avaliação, o artigo “The Use of Artificial Intelligence (AI) in Online Learning and Distance Education Processes: A Systematic Review of Empirical Studies” é classificado como:

\begin{enumerate}


    \item \textbf{Paradigma epistemológico:} Quali-Quanti \\
    O artigo adota uma abordagem em maior parte quantitativa, uma vez que se baseia em análise de literatura e técnicas de mineração de texto para analisar um grande volume de publicações existentes. Porém existem também aspectos qualitativos, já que o estudo busca interpretar os resultados e identificar tendências temáticas na pesquisa sobre inteligência artificial em educação à distância.
    
    \item \textbf{Tipo de método:} Dedutivo \\
    O método utilizado é dedutivo, pois parte de premissas gerais sobre o uso de inteligência artificial na educação a distância para analisar estudos específicos já realizados.
    
    \item \textbf{Objeto de pesquisa:} Bibliográfico \\
    A pesquisa é baseada em publicações já existentes sobre o tema, não envolvendo coleta de novos dados diretamente dos usuários ou ambientes de aprendizagem.
    
    \item \textbf{Profundidade:} Descritiva e Exploratória \\
    O estudo é descritivo ao apresentar um panorama geral sobre as pesquisas realizadas, incluindo algumas tendências temporais, áreas de maior interesse e temas emergentes. Também possui caráter exploratório, pois busca identificar novas áreas de investigação e questionamentos relacionados ao uso de inteligência artificial na educação a distância.
    
    \item \textbf{Tempo decorrido:} Transversal \\
    A análise abrange um período amplo, desde o final da década de 1990 até 2022, permitindo observar a evolução do interesse e das tendências de pesquisa ao longo do tempo.
\end{enumerate}

\section{Query de Busca:} (\#6) AND ALL=(use of (artificial intelligence or ai) learning in education)

\section{5 Artigos mais relevantes na pesquisa:}

\begin{enumerate}
    \item Artificial Intelligence and Reflections from Educational Landscape: A Review of AI Studies in Half a Century ("Inteligência Artificial e Reflexões do Cenário Educacional: Uma Revisão dos Estudos de IA em Meio Século" em Português)

    \item Guest Editorial: Human-centered AI in Education: Augment Human Intelligence with Machine Intelligence ("Editorial convidado: IA centrada no ser humano na educação: aumente a inteligência humana com inteligência de máquina" em Português)
    
    \item The Use of Artificial Intelligence (AI) in Online Learning and Distance Education Processes: A Systematic Review of Empirical Studies ("O Uso da Inteligência Artificial (IA) em Processos de Aprendizagem Online e Educação a Distância: Uma Revisão Sistemática de Estudos Empíricos" em Português)
    
    \item Enhancing cardiovascular artificial intelligence (AI) research in the Netherlands: CVON-AI consortium ("Melhorando a pesquisa de inteligência artificial (IA) cardiovascular na Holanda: consórcio CVON-AI" em Português)
    
    \item Artificial Intelligence Literacy: Developing a Multi-institutional Infrastructure for AI Education ("Alfabetização em Inteligência Artificial: Desenvolvendo uma Infraestrutura Multiinstitucional para Educação em IA" em Português)
    
\end{enumerate}

\end{document}